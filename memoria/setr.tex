\documentclass[12pt,a4paper]{article}
\usepackage[latin1]{inputenc}
\usepackage{amsmath}
\usepackage{amsfonts}
\usepackage{amssymb}
\usepackage[left=2cm]{geometry}
\usepackage[spanish]{babel}
\usepackage{graphicx}
%\addto\captionsspanish{ \renewcommand{\chaptername}{Parte}}

\title{SETR}
\begin{document}

\begin{titlepage}
\setlength{\parindent}{0pt}
\thispagestyle{empty}
\newgeometry{left=2cm, top=3cm}
\begin{figure}[h]
\includegraphics[scale=0.5]{Img/logoURJC.png}
\end{figure}
 {\Large{Escuela T�cnica Superior en Ingenier�a Inform�tica}}
\vfill
\begin{center}
	\hrulefill \\
	{\Huge{\textbf{Sistemas Empotrados y de Tiempo Real}}\\\huge{Grupo 10: Juego de Mesa}}\\
	\hrulefill
\end{center}

\vfill
\begin{flushright}
{\large 	\begin{tabular}{lll}
	
    GII+GIS & : & Jes�s Alcalde Alc�zar\\
    		&   & Adri�n Guti�rrez Jim�nez\\
    GII+MAT	& : & Javier Jim�nez del Peso\\
    
	\end{tabular}}
\end{flushright}
\newpage
\end{titlepage}

\tableofcontents
\setlength{\parindent}{0pt}
\newpage

\part{Introducci�n}
\section{Idea inicial}
\section{Funcionamiento del juego}
...
\part{Desarrollo hardware}
\section{Problemas encontrados y soluciones}
\section{I2C Expander/EEPROM}
\section{Esquema el�ctrico}
\section{Dise�o de piezas para impresora 3D}
...
\part{Desarrollo software}
\section{Dise�o}
...
\part{Conclusi�n}
\section{Posibles mejoras}
\section{Opini�n personal}

\end{document}